\chapter{Introduction}


\section{Motivation}


As applications move from the desktop to the web browser security needs to be taken into consideration. The Web Cryptography API provides a Javascript API for web applications to perform basic cryptographic operations and key management functions. The Web Crypto API is in the editor's draft stage of the standards recommendation process.\cite{webcrypto-overview} Despite the infancy and status of the Web Crypto API browsers have began implementing it. The Chromium browser project has completed their implementation of the Web Crypto API.\cite{webcrypto-chromium} This is the first API for browser based Javascript that implements cryptographic operations in native code. Prior to the WebCrypto API all Javascript cryptography was performed by 3rd party Javascript libraries such as the Stanford Javascript Crypto Library.\cite{sjcl-library} This is a major improvement in the maturity of the browser for becoming a platform for security and privacy focused applications. The exploration of this young API is important to determine its capabilities and evaluate its interfaces.


\section{Our Contribution}


At Cal Poly the theme of each department's educational goals is "learn by doing". In this thesis that was the goal. The Web Crypto API specification has a chapter on use cases. One of these use cases is secure messaging. The API says "A web application may wish to employ message layer security using schemes such as off-the-record (OTR) messaging, even when these messages have been securely received, such as over TLS."\cite{webcrypto-overview} This thesis aims to test the Webcrypto API by doing exactly that. The OTR messaging protocol is implemented in Typescript by using the Web Crypto API for all of the cryptographic operations and key management functionality that OTR requires. This work is novel as it uses the new WebCrypto API to implement a fully functional end to end encrypted messaging application.  


The relevant background concepts used to implement the OTR messaging application are reviewed in chapter 2. Related work is covered in chapter 3. The implementation itself is detailed in chapter 4. The validation of how the implementation was tested is examined in chapter 5. Future work that could be done to the implementation is discussed in chapter 6. Finally the conclusion is in chapter 7.
  

